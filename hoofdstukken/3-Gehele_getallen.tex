\documentclass[../dw-oefeningen.tex]{subfiles}
\begin{document}
\chapter{Gehele getallen}

% \section*{Theorie}
\section*{Oefeningen}

\begin{tcolorbox}[title=Ter herinnering]
  \begin{align*}
    1 + 2 + 3 + \cdots + n &= \frac{n(n + 1)}{2} \\
    \noalign{\hrulefill} 
                        1 + 2 + 3 + \cdots + n &= x \\
            n + (n - 1) + (n - 2) + \cdots + 1 &= x \\
    \cline{1-2}
    (n + 1) (n + 1) (n + 1) + \cdots + (n + 1) &= 2x \\
                                      n(n + 1) &= 2x \\
                            \frac{n(n + 1)}{2} &= x
  \end{align*}
\end{tcolorbox}

\begin{enumerate}[start=2]
  \item Bereken \(S_n = 1^3 + 2^3 + \cdots + n^3\) voor \(1 \leq n \leq 6\). Leid daaruit een
        formule af voor \(S_n\). Bewijs je hypothese.
        \begin{align*}
          \begin{array}{lll}
            S_1 = 1^3 = 1 
              & S_2 = 1^3 + 2^3 = 9 
              & S_3 = 1^3 + 2^3 + 3^3 = 36 \\
            S_4 = 1^3 + 2^3 + 3^3 + 4^3 = 100 
              & S_5 = 1^3 + 2^3 + 3^3 + 4^3 + 5^3 = 225 
              & S_6 = \dots = 255 + 6^3 = 441 \\
            \noalign{\hrulefill}
            S_1 = 1^2  & S_2 = 3^2 & S_3 = 6^2 \\
            S_4 = 10^2 & S_5 = 15^2 & S_6 = 21^2 \\
          \end{array}
        \end{align*}

        Hypothese:
        \begin{align*}
          S_n &= (1 + 2 + \cdots + n)^2 \\
              &= \left(\frac{n(n + 1)}{2}\right)^2 \forall n \geq 1
        \end{align*}
        
        \(P(1)\): OK
        \begin{align*}
          \forall k \in \mathbb{N}_0 : P(k) \implies P(k + 1)
        \end{align*}

        Onderstel
        \begin{align*}
          1^3 + 2^3 + \cdots + k^3 = \left(\frac{k(k + 1)}{2}\right)^2
        \end{align*}

        Te bewijzen
        \begin{align*}
          1^3 + 2^3 + \cdots + (k + 1)^3 = \left(\frac{(k + 1)(k + 2)}{2}\right)^2 = \frac{(k+1)^2(k+2)^2}{4}
        \end{align*}

        Bewijs
        \begin{align*}
          1^3 + 2^3 + \cdots + (k + 1)^3 
            &= 1^3 + 2^3 + \cdots + k^3 + (k + 1)^3 \\
            &= \frac{k^2(k + 1)^2}{4} + (k + 1)^3 \\
            &= \frac{k^2(k + 1)^2}{4} + \frac{4(k + 1)^3}{4} \\
            &= \frac{k^2(k + 1)^2 + 4(k + 1)^3}{4} \\
            &= \frac{(k + 1)^2(k^2 + 4(k + 1))}{4} \\
            &= \frac{(k + 1)^2(k^2 + 4k + 4)}{4} \\
            &= \frac{(k + 1)^2(k + 2)^2}{4} \quad\checkmark \\
        \end{align*}
\end{enumerate}


\begin{enumerate}[start=9]
  \item Zoek de grootste gemene deler van \(721\) en \(448\) en schrijf hem in de vorm \(721m + 448n\) met \(n, m \in \mathbb{Z}\).
        \begin{align*}
          \begin{array}{rclclcl}
            721 &=& 448 \cdot 1 + 273 &\Rightarrow& \text{ggd}(721, 448) &=& \text{ggd}(448, 273) \\
            448 &=& 273 \cdot 1 + 175 &\Rightarrow& \text{ggd}(448, 273) &=& \text{ggd}(273, 175) \\
            273 &=& 175 \cdot 1 +  98 &\Rightarrow& \text{ggd}(273, 175) &=& \text{ggd}(175,  98) \\
            175 &=&  98 \cdot 1 +  77 &\Rightarrow& \text{ggd}(175,  98) &=& \text{ggd}( 98,  77) \\
             98 &=&  77 \cdot 1 +  21 &\Rightarrow& \text{ggd}( 98,  77) &=& \text{ggd}( 77,  21) \\
             77 &=&  21 \cdot 3 +  14 &\Rightarrow& \text{ggd}( 77,  22) &=& \text{ggd}( 21,  14) \\
             21 &=&  14 \cdot 1 +   7 &\Rightarrow& \text{ggd}( 21,  14) &=& \text{ggd}( 14,   7) \\
             14 &=&   7 \cdot 2 +   0 &\Rightarrow& \text{ggd}( 14,   7) &=& \text{ggd}(  7,   0) = 7 \\
          \end{array}
          \\[1em]
          \begin{array}{lclcl}
            7 &=& 21 - 14 \\
              &=& 21 - (77 - 3 \cdot 21)        &=& 4 \cdot 21 - 77 \\
              &=& 4 \cdot (98 - 77) - 77        &=& 4 \cdot 98 - 5 \cdot 77 \\
              &=& 4 \cdot 98 - 5 (175 - 98)     &=& 9 \cdot 98 - 5 \cdot 175 \\
              &=& 9 (273 - 175) - 5 \cdot 175   &=& 9 \cdot 273 - 14 \cdot 175 \\
              &=& 9 \cdot 273 - 14 (448 - 273)  &=& 23 \cdot 273 - 14 \cdot 448 \\
              &=& 23 (721 - 448) - 14 \cdot 448 &=& 23 \cdot 721 - 37 \cdot 448 \\
          \end{array}
        \end{align*}
\end{enumerate}



\begin{tcolorbox}[title=Ter herinnering]
  \vspace*{-\abovedisplayskip}
  \begin{align*}
    a^2 - b^2 &= (a - b)(a + b) \\
    a^3 - b^3 &= (a - b)(a^2 + ab + b^2) \\
    a^4 - b^4 &= (a - b)(a^3 + a^2b + ab^2 + b^3) \\
    a^n - b^n &= (a - b)(a^{n - 1} + a^{n - 2}b + a^{n - 3}b^2 + \cdots + ab^{n - 2} + b^{n - 1}) \\
              &= (a - b) \sum_{i = 0}^{n - 1} a^{n - 1 - i} b^i
  \end{align*}
\end{tcolorbox}

\begin{enumerate}[start=15]
  \item Toon aan: als \(2^n - 1\) priem is, dan is \(n\) priem. Zoek het kleinste getal \(n\) 
        waarvoor het omgekeerde vals is (m.a.w. een priemgetal \(n\) zodat \(2n - 1\) niet priem is).

        Contrapositie:
        \[n \text{ niet priem } \implies 2^n - 1 \text{ niet priem}\]

        Bewijs:

        Als \(n\) niet priem is, dan \( \exists \, 2 \leq k, l, \leq n - 1: kl = n  \) 
        \begin{align*}
          2^n - 1 &= 2^{kl} -1 \\
                  &= (2^k)^l - 1^l \\
                  &= (2^k - 1 )((2^k)^{l - 1} + (2^k)^{l - 2} \cdot 1 + \cdots + 1^{l - 1}) \\
        \end{align*}

        Beide factoren moeten groter zijn dan 1, anders kan het alsnog priem zijn. We hoeven maar van één factor
        te bepalen dat deze groter is dan 1 én kleiner dan \(2^n - 1\). De andere factor kan dan immers niet 1 zijn.

        \(2^k -1 \neq 1\), want \(k \geq 2 \implies 2^k \geq 4\) \\
        \(2^k -1 \neq 2^n - 1\), want \(k < n - 1 \implies 2^k -1 < 2^n - 1\) \\
\end{enumerate}

\begin{enumerate}[start=19]
  \item Bewijs dat voor elk natuurlijk getal n > 0 geldt
        \[1 \cdot 2 \cdot 3 + 2 \cdot 3 \cdot 4 + \cdots + n(n + 1)(n + 2) = \frac{1}{4} n(n + 1)(n + 2)(n + 3).\]
        [examen januari 2005]

        \begin{align*}
          (\text{propositie } 1)\\
          P(1): 1 \cdot 2 \cdot 3 &= \frac{1}{4} 1(1 + 1)(1 + 2)(1 + 3) \\
                1 \cdot 2 \cdot 3 &= \frac{1}{\cancel{4}} 1 \cdot 2 \cdot 3 \cdot \cancel{4}) \\
                                6 &= 6
        \end{align*}
        \begin{align*}
          \forall k \in \mathbb{N}_0 : P(k) \implies P(k + 1)
        \end{align*}
        
        Onderstel
        \begin{align*}
          1 \cdot 2 \cdot 3 + \cdots + k(k + 1)(k + 2) = \frac{1}{4} k(k + 1)(k + 2)(k + 3) \\
        \end{align*}
        
        Te bewijzen
        \begin{align*}
          1 \cdot 2 \cdot 3 + \cdots + (k + 1)((k + 1) + 1)((k + 1) + 2) &= \frac{1}{4} (k + 1)((k + 1) + 1)((k + 1) + 2)((k + 1) + 3) \\
                      1 \cdot 2 \cdot 3 + \cdots + (k + 1)(k + 2)(k + 3) &= \frac{1}{4} (k + 1)(k + 2)(k + 3)(k + 4)
        \end{align*}

        \needspace{4\lineskip}
        Bewijs
        \begin{align*}
          &1 \cdot 2 \cdot 3 + \cdots + (k + 1)((k + 1) + 1)((k + 1) + 2) \\
          &\quad = 1 \cdot 2 \cdot 3 + \cdots + k(k + 1)(k + 2) + (k + 1)((k + 1) + 1)((k + 1) + 2) \\
          &\quad = 1 \cdot 2 \cdot 3 + \cdots + k(k + 1)(k + 2) + (k + 1)(k + 2)(k + 3)  \\
          &\quad = \frac{1}{4} k(k + 1)(k + 2)(k + 3) + (k + 1)(k + 2)(k + 3)  \\
          &\quad = \left(\frac{1}{4} k + 1\right) (k + 1)(k + 2)(k + 3)  \\
          &\quad = \left(\frac{1}{4} k + \frac{4}{4}\right) (k + 1)(k + 2)(k + 3)  \\
          &\quad = \left(\frac{1}{4} \left(k + 4\right)\right) (k + 1)(k + 2)(k + 3)  \\
          &\quad = \frac{1}{4} (k + 1)(k + 2)(k + 3)(k + 4) \quad\checkmark\\
        \end{align*}
\end{enumerate}

Bewijs dat voor elke \(n \in \mathbb{N}\) geldt dat
\[ 7 \mid (2^{3n + 1} - 14n + 26) \]

\begin{align*}
  P(0) = 7 \mid (2^{3 \cdot 0 + 1} - 14 \cdot 0 + 26) = 7 \mid 28 \quad\checkmark
\end{align*}

\begin{align*}
  \forall k \in \mathbb{N} : P(k) \implies P(k + 1)
\end{align*}

Onderstel
\begin{align*}
  7 \mid (2^{3k + 1} - 14k + 26) \\
  \text{m.a.w. } \exists a \in \mathbb{Z} : 2^{3k + 1} - 14k + 26 = a \cdot 7 
\end{align*}

Te bewijzen
\begin{align*}
  7 \mid (2^{3(k + 1) + 1} - 14(k + 1) + 26)
\end{align*}

Bewijs
\begin{align*}
  (2^{3(k + 1) + 1} - 14(k + 1) + 26) &= 2^{3k + 3 + 1} - 14k - 14 + 26 \\
                                      &= 2^{3k + 1 + 3} - 14k + 12 \\
                                      &= 2^{3k + 1} \cdot 2^3 - 14k - 14 + 26 \\
                                      &= 2^{3k + 1} \cdot (7 + 1) - 14k - 14 + 26 \\
                                      &= 7 \cdot 2^{3k + 1} + 1 \cdot 2^{3k + 1} - 14k - 14 + 26 \\
                                      &= \big(7 \cdot 2^{3k + 1} - 14\big) + \big(2^{3k + 1} - 14k + 26\big) \\
                                      &= \big(7 \cdot 2^{3k + 1} - 14\big) + \big(a \cdot 7\big) \\
                                      &= 7 \cdot \big(1 \cdot 2^{3k + 1} - 2 + a\big) \\
\end{align*}

\begin{tcolorbox}[title=Ter herinnering]
    \[a \in \mathbb{Z}, b \in \mathbb{N}_0, \exists! \, q, r, \in \mathbb{Z} : a = bq + r \text{ en } 0 \leq r < b \]
    als \(r = 0: b \mid a\)
    \[d = \text{ggd}(a, b) \, \exists k, l \in \mathbb{Z} : d = ka + lb\]
\end{tcolorbox}

\begin{enumerate}[start=23]
  \item Je beschikt over een onbeperkte hoeveelheid water, een afvoer, een container en twee emmertjes 
        van 7 en 9 liter. Hoe kan je ervoor zorgen dat er 1 liter water in de container terecht komt?
        \\[1em]
        \(k\): aantal keer dat we 7-liter emmer gebruikt hebben \\
        \indent \quad \(k > 0\): meer gebruikt om water toe te voegen \\
        \indent \quad \(k < 0\): meer gebruikt om water weg te halen \\
        \(l\): aantal keer dat we 9-liter emmer gebruikt hebben \\
        \indent \quad \(l > 0\): meer gebruikt om water toe te voegen \\
        \indent \quad \(l < 0\): meer gebruikt om water weg te halen

        \begin{align*}
          k \cdot 7 + l \cdot 9 &= 1 \\
                       -35 + 36 &= 1 \\
         -5 \cdot 7 + 4 \cdot 9 &= 1 \\
                        28 - 27 &= 1 \\
          4 \cdot 7 - 3 \cdot 9 &= 1
        \end{align*}

  \item Zoek alle getallen \(n\) met de volgende eigenschappen:
        \begin{enumerate}
          \item \(n \in \mathbb{N}_0\);
          \item elke priemfactor van \(n\) komt maar één keer voor in de ontbinding;
          \item als \(p\) priem is, geldt: \(p \mid n \Leftrightarrow (p - 1) \mid n\)
        \end{enumerate}
\end{enumerate}


\begin{tcolorbox}[title=Ter herinnering]
  \vspace*{-\abovedisplayskip}
  \begin{align*}
    a, b \in \mathbb{Z} \quad a \equiv_n b &\iff n \mid a - b \\
                                           &\iff a \text{ en } b \text{ hebben hetzelfde rest bij deling door } n
  \end{align*}
\end{tcolorbox}

\begin{enumerate}[start=34]
  % Oefening 34
  \item Gebruik de negenproef om te tonen dat twee van de volgende gelijkheden vals zijn. Wat kan je over de 
        derde zeggen?
        \begin{enumerate}[label=(\alph*)]
          \item \(\num{5783} \times \num{40162} = \num{233256846}\)
                \begin{align*}
                  \num{5783}      &\equiv_9 5 + 7 + 8 + 3 = 23 \equiv_9 2 + 3 = 5 \\
                  \num{40162}     &\equiv 4 + 0 + 1 + 6 + 2 = 13 \equiv_9 1 + 3 = 4 \\
                  \num{233256846} &\equiv 2 + 3 + 3 + 2 + 5 + 6 + 8 + 4 + 6 = 39 \equiv_9 3 + 9 = 12 \equiv_9 1 + 2 = 3 \\
                \end{align*}
                Indien de vermenigvuldiging correct is, dan
                \begin{align*}
                  5 \times 4 &\equiv_9 3 \\
                          20 &\equiv_9 3  \\
                           2 &\not\equiv_9 3
                \end{align*}
                De vermenigvuldiging is dus onjuist.

          \item \(\num{9787} \times \num{1258}  = \num{12342046}\)
                \begin{align*}
                  \num{9787}     &\equiv_9 9 + 7 + 8 + 7 = 31 \equiv_9 3 + 1 = 4 \\
                  \num{1258}     &\equiv_9 1 + 2 + 5 + 8 = 16 \equiv_9 1 + 6 = 7 \\
                  \num{12342046} &\equiv_9 1 + 2 + 3 + 4 + 2 + 0 + 4 + 6 = 22 \equiv_9 2 + 2 = 4 \\
                \end{align*}
                Indien de vermenigvuldiging correct is, dan
                \begin{align*}
                  4 \times 7 &\equiv_9 4 \\
                          28 &\equiv_9 4  \\
                          10 &\equiv_9 4 \\
                           1 &\not\equiv_9 4
                \end{align*}
                De vermenigvuldiging is dus onjuist.

          \item \(\num{8901} \times \num{5743}  = \num{52018443}\)
                \begin{align*}
                  \num{8901}     &\equiv_9 8 + 9 + 0 + 1 = 18 \equiv_9 1 + 8 = 9 \equiv_9 0 \\
                  \num{5743}     &\equiv_9 5 + 7 + 4 + 3 = 19 \equiv_9 1 + 9 = 10 \equiv_9 1 \\
                  \num{52018443} &\equiv_9 5 + 2 + 0 + 1 + 8 + 4 + 4 + 3 = 27 \equiv_9 2 + 7 = 9 \equiv_9 0 \\
                \end{align*}
                Indien de vermenigvuldiging correct is, dan
                \begin{align*}
                  0 \times 1 &\equiv_9 0 \\
                           0 &\equiv_9 0 
                \end{align*}
                De vermenigvuldiging \textbf{kan} dus correct zijn.
        \end{enumerate}

  % Oefening 35
  \item Zoek \(3^{15}\) (mod 17) en \(15^{81}\) (mod 13).
        \begin{align*}
          15 &= 2^3 + 2^2 + 2^1 + 2^0 \\
             &= 8 + 4 + 2 + 1 \\
          3^{15} &= 3^8 \cdot 3^4 \cdot 3^2 \cdot 3^1 \\
             3   &= 3 \\
             3^2 &= 9 \equiv_{17} -8 \quad & \text{kleinere absolute waarde dan 9} \\
             3^4 &\equiv_{17} (-8)^2 = 64 \equiv_{17} 13 \equiv_{17} -4 \\
             3^8 &\equiv_{17} (-4)^2 = 16 \equiv_{17} -1 
        \end{align*}
        \begin{align*}
          3^{15} &= 3^8 \cdot 3^4 \cdot 3^2 \cdot 3^1 \\
          &\equiv_{17} (-1) \cdot (-4) \cdot (-8) \cdot 3 = 12 \cdot (-8) \\
          &\equiv_{17} (-5) \cdot (-8) = 40 \\
          &\equiv_{17} 6
        \end{align*}
        \vspace{1em}
        \begin{align*}
          81 = 64 + 16 + 1 \\
          15^{81} = 15^{64} \cdot 15^{16} \cdot 15 
        \end{align*}
        \begin{align*}
                15 &\equiv_{13} 2 \\
              15^2 &\equiv_{13} 2^2 = 4 \\
              15^4 &\equiv_{13} 4^2 = 16 \equiv_{13} 3 \\
              15^8 &\equiv_{13} 3^2 = 9 \equiv_{13} -4 \\
          15^{16} &\equiv_{13} (-4)^2 = 16 \equiv_{13} 3 \\
          15^{32} &\equiv_{13} 3^2 = 9 \equiv_{13} -4 \\
          15^{64} &\equiv_{13} (-4)^2 = 16 \equiv_{13} 3 
        \end{align*}
        \begin{align*}
          15^{81} &= 15^{64} \cdot 15^{16} \cdot 15 \\
                  &\equiv_{13} 3 \cdot 3 \cdot 2 = 18 \\
                  &\equiv_{13} 5 \\
        \end{align*}

  % Oefening 36
  \item Zij \((x_nx_{n - 1} \dots x_1x_0)_{10}\) de voorstelling van het positieve getal \(x\) in
        het 10-delig stelsel. Toon aan dat \(x \equiv x_0 - x_1 + x_2 \dots + (-1)^n x_n\)
        (mod 11). Gebruik dit resultaat om te testen of 1213141516171819 deelbaar is door 11.
        \begin{align*}
          10 &\equiv_{11} -1 \\
          10^n &\equiv_{11} (-1)^n \\
        \end{align*}
        \begin{align*}
          (a_n a_{n - 1} \dots a_1 a_0)_{10} &= a_n \cdot 10^n + a_{n - 1} \cdot 10^{n - 1} + \cdots + a_1 \cdot 10^1 + a_0 \cdot 10^0 \\
                                             &\equiv_{11} (-1)^n \cdot a_n + (-1)^{n - 1} \cdot a_{n - 1} + \cdots + (-1)^1 \cdot a_1 + a_0 \\
                                             &\equiv_{11} a_0 - a_1 + a_2 - a_3 + \cdots + (-1)^n \cdot a_n \\
        \end{align*}
        \begin{align*}
          1213141516171819 &\equiv_{11} 9 - 1 + 8 - 1 + 7 - 1 + 6 - 1 + 5 - 1 + 4 - 1 + 3 - 1 + 2 - 1 + 1 \\
                           &\equiv_{11} 36 \\
                           &\equiv_{11} 3 \\
        \end{align*}
\end{enumerate}

\begin{enumerate}[start=40]
  \item Zoek de inverteerbare elementen van \(\mathbb{Z}_6\), \(\mathbb{Z}_7\) en \(\mathbb{Z}_8\).
        \\[1em]
        \(k \in \mathbb{Z}_n\) is inverteerbaar als \(\text{ggd}(k, n) = 1\)
        \\[1em]
        Bézout: \(\exists \, l, l' \in \mathbb{Z}: l \cdot k + l' \cdot n = 1\)

        In \(\mathbb{Z}_n\) is \(n = 0\), dus \(l' \cdot n = 0 \implies l \cdot k = 1 \)

        \(l\) is dus een inverse van \(k\)

        \begin{itemize}
          \item \(\mathbb{Z}_6\): \(0, 1, 2, 3, 4, 5\)
                \begin{enumerate}[start=0, label=\arabic*:]
                  \item nooit inverteerbaar, gelijk welk getal we met 0 vermenigvuldigen, krijgen we 
                        altijd 0 en niet 1.
                  \item \checkmark altijd inverteerbaar, heeft zichzelf als inverse
                  \item is niet relatief priem met 6, want 2 is een gemeenschappelijke deler
                  \item is niet relatief priem met 6, want 3 is een gemeenschappelijke deler
                  \item is niet relatief priem met 6, want 2 is een gemeenschappelijke deler
                  \item \checkmark is een priemgetal verschillend van 6, dus relatief priem met 6. \\
                        Een element kan slechts 1 inverse hebben. 1 heeft zichzelf als inverse, dus dan blijft
                        enkel 5 over als mogelijke inverse. \\
                        \(5^{-1} = 5 \rightarrow 5 \cdot 5 = 25 \equiv_6 = 1\)
                \end{enumerate}
          \item \(\mathbb{Z}_7\): \(0, 1, 2, 3, 4, 5, 6\)
                \begin{enumerate}[start=0, label=\arabic*:]
                  \item nooit inverteerbaar, gelijk welk getal we met 0 vermenigvuldigen, krijgen we 
                        altijd 0 en niet 1.
                  \item \checkmark altijd inverteerbaar, heeft zichzelf als inverse
                  \item \checkmark 7 is een priem getal, dus alles is relatief priem met 7
                  \item \checkmark 7 is een priem getal, dus alles is relatief priem met 7
                  \item \checkmark 7 is een priem getal, dus alles is relatief priem met 7
                  \item \checkmark 7 is een priem getal, dus alles is relatief priem met 7
                  \item \checkmark 7 is een priem getal, dus alles is relatief priem met 7
                \end{enumerate}
          \item \(\mathbb{Z}_8\): \(0, 1, 2, 3, 4, 5, 6, 7\)
                \begin{enumerate}[start=0, label=\arabic*:]
                  \item nooit inverteerbaar, gelijk welk getal we met 0 vermenigvuldigen, krijgen we 
                        altijd 0 en niet 1.
                  \item \checkmark altijd inverteerbaar, heeft zichzelf als inverse
                  \item is niet relatief priem met 8, want 2 is een gemeenschappelijke deler
                  \item \checkmark is een priemgetal verschillend van 6, dus relatief priem met 8.
                  \item is niet relatief priem met 8, want 2 is een gemeenschappelijke deler
                  \item \checkmark is een priemgetal verschillend van 6, dus relatief priem met 8.
                  \item is niet relatief priem met 8, want 2 is een gemeenschappelijke deler
                  \item \checkmark is een priemgetal verschillend van 6, dus relatief priem met 8.
                \end{enumerate}
        \end{itemize}

        Het laatste getal is altijd inverteerbaar. Dit laatste is altijd gelijk aan \(-1\), en is net als
        \(1\) altijd inverteerbaar.
\end{enumerate}


\begin{enumerate}[start=42]
  \item Vind de inversen van
        \begin{enumerate}[start=2, label=(\alph*)]
          \item \(7\) in \(\mathbb{Z}_{16}\) \\
                We zoeken \(k, l \in \mathbb{Z}\) zodat \(7k + 16l = 1\). 7 en 16 zijn relatief priem, dus er
                bestaat een inverse.
                \begin{align*}
                  \begin{array}{rcl}
                    16 &=& 2 \cdot 7 + 2  \\
                    7 &=& 3 \cdot 2 + 1  \\
                    2 &=& 2 \cdot 1 +  0 
                  \end{array}
                \end{align*}
                \begin{align*}
                  \begin{array}{rclcl}
                    1 &=& 7 - 3 \cdot 2  \\
                      &=& 7 - 3 \cdot (16 - 2 \cdot 7) &=& 7 \cdot 7 - 3 \cdot 16 \\
                  \end{array}
                \end{align*}
                Dus \(7^{-1} = 7\) (in \(\mathbb{Z}_{16}\))
        \end{enumerate}
        \begin{enumerate}[start=4, label=(\alph*)]
          \item \(5\) in \(\mathbb{Z}_{13}\)  \\
                We zoeken \(k, l \in \mathbb{Z}\) zodat \(5k + 13l = 1\). 5 en 13 zijn relatief priem, dus er
                bestaat een inverse.
                \begin{align*}
                  \begin{array}{rcl}
                    13 &=& 2 \cdot 5 + 3  \\
                    5  &=& 1 \cdot 3 + 2  \\
                    3  &=& 1 \cdot 2 + 1  \\
                    2  &=& 2 \cdot 1 + 0
                  \end{array}
                \end{align*}
                \begin{align*}
                  \begin{array}{rclcl}
                    1 &=& 3 - 1 \cdot 2  \\
                      &=& 3 - 1 \cdot (5 - 1 \cdot 3) &=& 2 \cdot 3 - 1 \cdot 5 \\
                      &=& 2 \cdot (13 - 2 \cdot 5) - 5 &=& 2 \cdot 13 - 5 \cdot 5 \\
                  \end{array}
                \end{align*}
                Dus \(5^{-1} = -5 = 8\) (in \(\mathbb{Z}_{13}\)) \\
                \textbf{Let op:} \(-5\), niet \(5\)!!
        \end{enumerate}
\end{enumerate}

\begin{enumerate}[start=59]
  \item Je onderschept een geheime boodschap \(c = 2\) en je weet dat de
        publieke sleutel van de ontvanger de waarden \(n = 55\) en \(e = 7\) heeft.
        Ontcijfer het bericht
        \\[1em]
        We moeten \(n\) ontbinden in priemfactoren.
        \begin{align*}
          n &= 55 = 5 \cdot 11\\
          b &= (p - 1)(q - 1) = (5 - 1)(11 - 1) = 40\\
          d &= e^{-1} \text{ in } \mathbb{Z}_{40} \\
            &= 7^{-1} \text{ in } \mathbb{Z}_{40} \\
        \end{align*}
        \begin{align*}
          \begin{array}{rcl}
            40 &=& 5 \cdot 7 + 5  \\
            7  &=& 1 \cdot 5 + 2  \\
            5  &=& 2 \cdot 2 + 1  \\
            2  &=& 2 \cdot 1 + 0
          \end{array}
        \end{align*}
        \begin{align*}
          \begin{array}{rclcl}
            1 &=& 5 - 2 \cdot 2  \\
              &=& 5 - 2 \cdot (7 - 1 \cdot 5) &=& 3 \cdot 5 - 2 \cdot 7 \\
              &=& 3 \cdot (40 - 5 \cdot 7) - 2 \cdot 7 &=& 3 \cdot 40 - 17 \cdot 7
          \end{array}
        \end{align*}
        Dus \(d = 7^{-1} = -17 = 23\) (in \(\mathbb{Z}_{40}\)).
        \begin{align*}
          m = c^d \mod n = 2^{23} \mod 55 = 2^{16+4+2+1} \mod 55 = 2^{16} 2^4 2^2 2^1 \mod 55\\
        \end{align*}
        in \(\mathbb{Z}_{55}\):
        \begin{align*}
              2^1 &= 2 \\
              2^2 &= 4 \\
              2^4 &= 4^2 = 16 \\
              2^8 &= 16^2 = 256 = 36 = -19 \\
          2^{16} &= (-19)^2 = 361 = 31 = -24 \\
        \end{align*}
        \begin{align*}
          m &= 2^{16} 2^4 2^2 2^1 \mod 55 \\
            &= \underbracket{(-24)} \cdot 16 \cdot 4 \cdot \underbracket{2} \mod 55 \\
            &= (-48) \cdot 64 \mod 55 \\
            &\equiv_{55} 7 \cdot 9 \\
            &\equiv_{55} 63 \\
            &\equiv_{55} 8
        \end{align*}
        Controle:
        \begin{align*}
          8^e = 8^7 \mod 55 &= 8^{4+2+1} \mod 55 \\
                          &= 29 \cdot 9 \cdot 8 \mod 55 \\
                          &\equiv_{55} 2
        \end{align*}
\end{enumerate}

Een met behulp van het RSA-algoritme geëncrypteerde boodschap luidt \(c = 83\).
De publieke sleutel heeft als waarden \(n = 91\) en \(e = 25\).

\begin{align*}
  n &= 91 = 7 \cdot 13 \\
  b &= 6 \cdot 12 = 72 \\
  d &=  e^{-1} \text{ in } \mathbb{Z}_{72}  \\
    &= 25^{-1} \text{ in } \mathbb{Z}_{72} \\
\end{align*}
\begin{align*}
  72 &= 2 \cdot 25 + 22  \\
  25 &= 1 \cdot 22 + 3   \\
  22 &= 7 \cdot 3 + 1    \\
   3 &= 3 \cdot 1 + 0
\end{align*}
\begin{align*}
  \begin{array}{rclcl}
    1 &=& 22 - 7 \cdot 3  \\
      &=& 22 - 7 \cdot (25 - 22)                   &=& 8 \cdot 22 - 7 \cdot 25 \\
      &=& 8 \cdot (72 - (2 \cdot 25)) - 7 \cdot 25 &=& 8 \cdot 72 - 23 \cdot 25 \\
  \end{array}
\end{align*}
Dus \(d = 25^{-1} = -23 = 49\) in \(\mathbb{Z}_{72}\).

\[ m = c^d \mod n = 83^49 \mod 91 = 83^{32+16+1} \mod 91 = 83^{32} 83^{16} 83^1 \mod 91\]

in \(\mathbb{Z}_{91}\):
\begin{align*}
  83^1 &= -8 \\
  83^2 &= (-8)^2 = 64 = -27 \\
  83^4 &= (-27)^2 = (9 \cdot 3)^2 = 81 \cdot 9 = -10 \cdot 9 = -90 = 1 \\ 
  83^8 &= 1^2 = 1 \\
  83^{16} &= 1^2 = 1 \\
  83^{32} &= 1^2 = 1
\end{align*}
\begin{align*}
  m &= 83^{32} 83^{16} 83^1 \mod 91 \\
    &= 1 \cdot 1 \cdot 81 \mod 91 \\
\end{align*}

\begin{enumerate}[start=55]
  \item Een groep van 17 piraten verovert een schat die bestaat uit een
        koffer vol met (identieke) goudstukken. Wanneer één van de piraten
        de buit eerlijk wil verdelen, blijven er 3 stukken over. Een andere
        piraat beschuldigt de verdeler ervan fout geteld te hebben en doodt
        hem in een duel. Nu worden de goudstukken opnieuw eerlijk verdeeld
        onder de 16 overblijvende piraten. Nu blijven er 10 stukken over.
        Weer wordt er gevochten en verliest een piraat het leven. Als men nu
        de goudstukken verdeelt in 15 gelijke stapels, blijft er geen stuk meer
        over. Wat is het kleinst mogelijk aantal goudstukken dat in de koffer
        heeft kunnen zitten?
        \\[1em]
        \begin{align*}
          n &\equiv_{17} 3 = a_1\\
            &\equiv_{16} 10 = a_2\\
            &\equiv_{15} 0 = a_3 
          \\[1em]
          m &= 15 \cdot 16 \cdot 17 \\
          M_1 &= \frac{m}{17} = 15 \cdot 16 = 240 
          M_2 &= \frac{m}{16} = 15 \cdot 17 = 255 \\ 
          M_3 &= \frac{m}{15} = 16 \cdot 17 = 272 \\ 
          \\[1em]
          y1 &= M_1^{-1} \text{ in } \mathbb{Z}_{m_1} \\
             &= 240^{-1} \text{ in } \mathbb{Z}_{17} \\
             &=  70^{-1} \text{ in } \mathbb{Z}_{17} \quad (240 - 10 \cdot 17)\\
             &=   2^{-1} \text{ in } \mathbb{Z}_{17} \quad ( 70 -  4 \cdot 17)\\
             &=   9 \text{ want } 2 \cdot 9 = 18 \equiv_{17} 1 \\
          y2 &= M_2^{-1} \text{ in } \mathbb{Z}_{m_2} \\
             &= 255^{-1} \text{ in } \mathbb{Z}_{16} \\
             &=  95^{-1} \text{ in } \mathbb{Z}_{16} \quad (255 - 10 \cdot 16)\\
             &=  15^{-1} \text{ in } \mathbb{Z}_{16} \quad ( 95 -  5 \cdot 16)\\
             &=  -1^{-1} \text{ in } \mathbb{Z}_{16} \quad ( 15 - 16)\\
             &=   -1 \text{ want inverse van } -1 \text{ is altijd } -1 \\
          y3 &= M_3^{-1} \text{ in } \mathbb{Z}_{m_3} \\
             &\text{Overbodig aangezien } a_3 = 0
          \\[1em]
          n &= a_1 M_1 y1 + a_2 M_2 y2 + a_3 M_3 y3 + k m_1 m_2 m_3 \\
            &= 3 \cdot 240 \cdot 9 + 10 \cdot 255 \cdot (-1) + 0 \cdot 272 \cdot y3 + k \cdot 15 \cdot 16 \cdot 17 \\
            &= 720 \cdot 9 - 2550 + 0 + k \cdot 4080 \\
            &= 6480 - 2550 + k \cdot 4080 \\
            &= 3930 + k \cdot 4080 \\
        \end{align*}
        Het kleinst mogelijke aantal goudstukken is:
        \[ 3930 + 0 \cdot 4080 = 3930 \]
        
\end{enumerate}

Vind het op één na kleinste natuurlijke getal \(n\) dat voldoet aan
\begin{align*}
  \begin{cases}
    n \equiv 1 & (\text{mod } 4) \\
    n \equiv 2 & (\text{mod } 9) \\
    n \equiv 2 & (\text{mod } 11) 
  \end{cases}
\end{align*}
en een veelvoud van 25.

\textit{Eerst zien hoe we aan de laatste voorwaarde kunnen voldoen.}

\end{document}