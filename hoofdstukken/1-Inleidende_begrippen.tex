\documentclass[../dw-oefeningen.tex]{subfiles}
\begin{document}
\chapter{Inleidende begrippen}

\begin{enumerate}
  % Oefening 1
  \item \(p(x) = ``x \text{ is deelbaar door 10}''\) en \(q(x) = ``x \text{ is deelbaar door 2}''\)
    \begin{enumerate}[label=\alph*)]
        \item Met symbolen (voor \(x\in\mathbb{Z}\)):
            \begin{itemize}
        \item \(p(x): x \mod 10 = 0\) (er bestaat \(k\in\mathbb{Z}\) zodat \(x=10k\))
        \item \(q(x): x \mod 2 = 0\) (er bestaat \(m\in\mathbb{Z}\) zodat \(x=2m\))
            \end{itemize}
        \item Negatie: \(\neg p(x)\): \(x \mod 10 \ne 0\) ("\(x\) is niet deelbaar door 10").
        \item Conjunctie: \(p(x) \land q(x)\): \(x \mod 10 = 0 \land x \mod 2 = 0\). Dit reduceert tot \(x \mod 10 = 0\) omdat \(x \mod 10 = 0 \impl x \mod 2 = 0\).
        \item Implicatie \(q(x)\impl p(x)\): \(x \mod 2 = 0 \Rightarrow x \mod 10 = 0\). \\
                Contrapositie: \(x \mod 10 \ne 0 \Rightarrow x \mod 2 \ne 0\).
        \item Equivalentie \(p(x)\biimpl q(x)\): \(x \mod 10 = 0 \Leftrightarrow x \mod 2 = 0\).
    \end{enumerate}
    Waarheidswaarden:
    \begin{itemize}
        \item \(q(x)\impl p(x)\) is \textbf{onwaar}: neem \(x=2\) dan \(2 \mod 2 = 0\) maar \(2 \mod 10 \ne 0\).
        \item Contrapositie \(\neg p(x)\impl \neg q(x)\) is dus ook onwaar: neem \(x=2\) dan antecedent \(2 \mod 10 \ne 0\) waar, consequent \(2 \mod 2 \ne 0\) onwaar.
        \item \(p(x)\land q(x)\) is waar precies voor \(x \mod 10 = 0\) (alle veelvouden van 10).
        \item Negatie \(\neg p(x)\) is waar voor alle \(x\) met \(x \mod 10 \ne 0\).
        \item Equivalentie \(p(x)\biimpl q(x)\) is onwaar: \(x=2\) is tegenvoorbeeld (rechts waar, links onwaar); ook \(x=10\) maakt beide waar maar één voorbeeld volstaat om onwaarheid aan te tonen.
    \end{itemize}

  % Oefening 2
  \item Waarheidstafel voor de exclusieve of (Xor) \(p\xor q\):
  \[
  \begin{array}{c c | c}
    p & q & p\oplus q\\\hline
    T & T & F\\
    T & F & T\\
    F & T & T\\
    F & F & F
  \end{array}
  \]
  Een equivalente uitdrukking: \((p\lor q) \land \neg(p\land q)\) of ook \((p\lor q) \land (\neg p \lor \neg q)\) of in som-van-producten: \((p\land \neg q) \lor (\neg p \land q)\).

  % Oefening 3
  \item Waarheidstabellen:
  \begin{enumerate}[label=\roman*)]
  \item \((p \impl q) \impl (q \impl p)\):
     \[
     \begin{array}{c c | c c | c}
       p & q & p\impl q & q\impl p & (p\impl q)\impl(q\impl p) \\\hline
       T & T & T & T & T \\
       T & F & F & T & T \\
       F & T & T & F & F \\
       F & F & T & T & T \\
     \end{array}
     \]

  \item \(q \biimpl (\neg p \lor \neg q)\):
     \[
     \begin{array}{c c | c c | c}
       p & q & \neg p \lor \neg q & q\biimpl(\neg p \lor \neg q) \\\hline
       T & T & F & F \\
       T & F & T & F \\
       F & T & T & T \\
       F & F & T & F \\
     \end{array}
     \]

  \item \([(p\impl q) \land (q\impl r)] \impl (p \impl r)\) (transitiviteit van implicatie):
     \[
     \begin{array}{c c c | c c | c | c | c}
       p & q & r & p\impl q & q\impl r & (p\impl q)\land(q\impl r) & p\impl r & \text{geheel} \\\hline
       T & T & T & T & T & T & T & T \\
       T & T & F & T & F & F & F & T \\
       T & F & T & F & T & F & T & T \\
       T & F & F & F & T & F & F & T \\
       F & T & T & T & T & T & T & T \\
       F & T & F & T & F & F & T & T \\
       F & F & T & T & T & T & T & T \\
       F & F & F & T & T & T & T & T \\
     \end{array}
     \]
     Dit is een tautologie.
  \end{enumerate}
  % Oefening 4
  \item Toon aan dat \(\neg(p \lor q)\) en \(\neg p \land \neg q\) logisch equivalent zijn. Wat kan je zeggen over \(\neg(p \land q)\) en \(\neg p \lor \neg q\)?
  \begin{enumerate}[label=\alph*)]
    \item Waarheidstabel voor \(\neg(p \lor q)\) en \(\neg p \land \neg q\):
    \[
    \begin{array}{c c | c | c c | c}
      p & q & p \lor q & \neg(p \lor q) & \neg p \land \neg q & gelijk? \\\hline
      T & T & T & F & F & T \\
      T & F & T & F & F & T \\
      F & T & T & F & F & T \\
      F & F & F & T & T & T \\
    \end{array}
    \]
    De kolom \(\neg(p \lor q)\) is identiek aan \(\neg p \land \neg q\), dus ze zijn logisch equivalent (De Morgan).
    
    \item Waarheidstabel voor \(\neg(p \land q)\) en \(\neg p \lor \neg q\):
    \[
    \begin{array}{c c | c | c c | c}
      p & q & p \land q & \neg(p \land q) & \neg p \lor \neg q & gelijk? \\\hline
      T & T & T & F & F & T \\
      T & F & F & T & T & T \\
      F & T & F & T & T & T \\
      F & F & F & T & T & T \\
    \end{array}
    \]
    Ook hier zijn de kolommen \(\neg(p \land q)\) en \(\neg p \lor \neg q\) identiek (tweede De Morgan).
  \end{enumerate}

  % Oefening 5
  \item Toon aan dat \(p \biimpl q\) en \((p \impl q) \land (q \impl p)\) logisch equivalent zijn.
    \[
    \begin{array}{c c | c | c c | c}
        p & q & p \biimpl q & p \impl q & q \impl p & (p \impl q) \land (q \impl p) \\\hline
        T & T & T & T & T & T \\
        T & F & F & F & T & F \\
        F & T & F & T & F & F \\
        F & F & T & T & T & T \\
    \end{array}
    \]
    De kolom \(p \biimpl q\) is identiek aan \((p \impl q) \land (q \impl p)\), dus ze zijn logisch equivalent.

  % Oefening 6
  \item Elke uitspraak in een waarheidstafel kan waar of onwaar zijn. Het aantal rijen van een waarheidstafel wordt bepaald door het aantal mogelijke combinaties 
        van de samenstellende uitspraken.

  % Oefening 7
  \item Schrijf de waarheidstafels op voor volgende logische uitspraken en leid er een equivalente vorm voor de uitspraak uit af:
  \begin{enumerate}[label=\alph*)]
    \item \(\neg(\neg p)\):
    \[
    \begin{array}{c | c | c}
       &  & \text{equiv} \\
      p & \neg(\neg p) & \mathbf{p} \\\hline
      T & T & T \\
      F & F & F \\
    \end{array}
    \]

    \item \(\neg(p \land q)\):
    \[
    \begin{array}{c c | c | c | c}
       &  &  &  & \text{equiv} \\
      p & q & p \land q & \neg(p \land q) & \mathbf{\neg p \lor \neg q} \\\hline
      T & T & T & F & F \\
      T & F & F & T & T \\
      F & T & F & T & T \\
      F & F & F & T & T \\
    \end{array}
    \]

    \item \(\neg(p \lor q)\):
    \[
    \begin{array}{c c | c | c | c}
       &  &  &  & \text{equiv} \\
      p & q & p \lor q & \neg(p \lor q) & \mathbf{\neg p \land \neg q} \\\hline
      T & T & T & F & F \\
      T & F & T & F & F \\
      F & T & T & F & F \\
      F & F & F & T & T \\
    \end{array}
    \]

    \item \(\neg(p \Leftarrow q)\):
    \[
    \begin{array}{c c | c | c | c}
       &  &  &  & \text{equiv} \\
      p & q & p \Leftarrow q & \neg(p \Leftarrow q) & \mathbf{q \land \neg p} \\\hline
      T & T & T & F & F \\
      T & F & T & F & F \\
      F & T & F & T & T \\
      F & F & T & F & F \\
    \end{array}
    \]

    \item \(\neg(p \Leftrightarrow q)\):
    \[
    \begin{array}{c c | c | c | c}
       &  &  &  & \text{equiv} \\
      p & q & p \Leftrightarrow q & \neg(p \Leftrightarrow q) & \mathbf{(p \land \neg q) \lor (\neg p \land q)} \\\hline
      T & T & T & F & F \\
      T & F & F & T & T \\
      F & T & F & T & T \\
      F & F & T & F & F \\
    \end{array}
    \]
  \end{enumerate}

  % Oefening 8
  \item 
    \begin{itemize}
      \item \(\neg(p \land (\neg p))\): \\
            een propositie kan niet zowel waar als onwaar zijn. De negatie ervan is dus altijd waar.
      \item \(p \lor (\neg p)\): \\
            een propositie is altijd ofwel waar ofwel onwaar.
      \item \((p \land p) \Leftrightarrow p\): \\
            \((T \land T) \Leftrightarrow T\) of \((F \land F) \Leftrightarrow F\), dus altijd waar.
      \item \((p \land q) \Leftrightarrow (q \land p)\): \\
            als \(p\) en \(q\) beide waar zijn, dan zijn \(q\) en \(p\) ook beide waar.
      \item \((p \lor (q \lor r)) \Leftrightarrow ((p \lor q) \lor r)\): \\
            als één van de proposities waar is, dan is deze waar ongeacht de haakjes
      \item \(\neg(\neg p) \Leftrightarrow p\): \\
            als een propositie waar is, is de negatie onwaar en de negatie daarvan terug waar.
      \item \(p \Rightarrow (q \Rightarrow p)\): 
            \[
            \begin{array}{c c | c | c}
              p & q & q \Rightarrow p & p \Rightarrow (q \Rightarrow p) \\\hline
              T & T & T & T \\
              T & F & T & T \\
              F & T & F & T \\
              F & F & T & T \\
            \end{array}
            \]
            Een propositie impliceert zichzelf.
      \item \(\neg p \Rightarrow (p \Rightarrow q)\): 
            \[
            \begin{array}{c c | c | c | c}
              p & q & \neg p & p \Rightarrow q & \neg p \Rightarrow (p \Rightarrow q) \\\hline
              T & T & F & T & T \\
              T & F & F & F & T \\
              F & T & T & T & T \\
              F & F & T & T & T \\
            \end{array}
            \]
            ???
      \item \((p \Rightarrow q) \lor (q \Rightarrow p)\): 
            \[
            \begin{array}{c c | c | c | c}
              p & q & p \Rightarrow q & q \Rightarrow p & (p \Rightarrow q) \lor (q \Rightarrow p) \\\hline
              T & T & T & T & T \\
              T & F & F & T & T \\
              F & T & T & F & T \\
              F & F & T & T & T \\
            \end{array}
            \]
            ???
    \end{itemize}

  % Oefening 9
  \item 
      \[
      \begin{array}{c c | c | c | c | c}
        p & q & q \Rightarrow p & p \Rightarrow (q \Rightarrow p) & p \Rightarrow q & (p \Rightarrow q) \Rightarrow p \\\hline
        T & T & T & T & T & T \\
        T & F & T & T & F & T \\
        F & T & F & T & T & F \\
        F & F & T & T & T & F \\
         &  & & \underbracket{\quad\quad} &  & \underbracket{\quad\quad}
      \end{array}
      \]
      Deze zijn dus niet equivalent.

  % Oefening 10
  \item Noteer volgende oefeningen met behulp van kwantoren. Bepaal eventueel of de bewering waar of vals is. Schrijf de negatie van de bewering op met kwantoren en met woorden.
  \begin{enumerate}[label=\alph*)]
    \item \enquote{Alle mensen zijn slim.}: \\
          \(\forall \text{mens}: \text{mens is slim}\) \\
          Jammer genoeg is niet niet waar. \\
          Negatie: \(\exists \text{mens}: \neg(\text{mens is slim})\)
    \item \enquote{Er zijn mensen die groot zijn.}: \\
          \(\exists \text{mens}: \text{mens is groot}\) \\
          Waar. \\
          Negatie : \(\not\exists \text{mens}: \text{mens is groot}\)
    \item \enquote{Er zijn mensen die groot zijn en lang haar hebben.} \\
          \(\exists \text{mens}: (\text{mens is groot} \land \text{mens heeft lang haar})\) \\
          Waar \\
          Negatie: \(\not\exists \text{mens}: (\text{mens is groot} \land \text{mens heeft lang haar})\) \\
          of: \(\forall \text{mens}: \neg(\text{mens is groot}) \lor \neg(\text{mens heeft lang haar})\)
    \item \enquote{Niet alle mensen hebben kort haar.} \\
          \(\exists \text{mens}: \neg(\text{mens heeft kort haar})\) \\
          Waar \\
          Negatie: \(\forall \text{mens}: \text{mens heeft kort haar}\)
    \item \enquote{Alle wegen leiden naar Rome.} \\
          \(\forall \text{weg}: \text{weg leidt naar Rome}\) \\
          Onwaar \\
          Negatie: \(\exists \text{weg}: \neg(\text{weg leidt naar Rome})\)
    \item \enquote{Voor elke mens geldt: als hij groot is, dan is hij niet klein.} \\
          \(\forall \text{mens}: (\text{mens is groot} \implies \neg(\text{mens is klein}))\) \\
          Waar \\
          Negatie: \(\exists \text{mens}: (\text{mens is groot} \implies \text{mens is klein})\)
    \item \enquote{Een geheel getal is positief.} \\
          \(\forall \text{geheel getal}: (\text{geheel getal is positief})\) \\
          Onwaar \\
          Negatie: \(\exists \text{geheel getal}: \neg(\text{geheel getal is positief})\)
    \item \enquote{Elk natuurlijk getal is even.} \\
          \(\forall \text{natuurlijk getal}: (\text{natuurlijk getal is even})\) \\
          Onwaar \\
          Negatie: \(\exists \text{natuurlijk getal}: \neg(\text{natuurlijk getal is even})\)
    \item \enquote{Sommige reële getallen zijn positief.} \\
          \(\exists \text{reëel getal}: (\text{reëel getal is positief})\) \\
          Waar \\
          Negatie: \(\not\exists \text{reëel getal}: (\text{reëel getal is positief})\)
  \end{enumerate}

  % Oefening 11
  \item Schrijf alle deelverzamelingen van \(\{1,2,3\}\).
  \[
    \emptyset, \{1\}, \{2\}, \{3\}, \{1,2\}, \{1,3\}, \{2,3\}, \{1,2,3\}
  \]
  
  % Oefening 12
  \item Hoeveel deelverzamelingen heeft een verzameling met 2 elementen? Met 3 elementen? Met \(n\) elementen?
    \begin{itemize}
      \item 2 elementen: \(2^2 = 4\) deelverzamelingen
      \item 3 elementen: \(2^3 = 8\) deelverzamelingen
      \item \(n\) elementen: \(2^n\) deelverzamelingen
    \end{itemize}

  % Oefening 13
  \item Wanneer behoort een element niet tot \(A \cap B\)? Vul aan: \(x \not\in A \cap B \Leftrightarrow\)
  \\
  Wanneer deze niet tot \(A\) of niet tot \(B\) behoort.
  \[
    x \not\in A \cap B \Leftrightarrow x \not\in A \lor x \not\in B
  \]

  % Oefening 14
  \item Analoog: \(x \not\in A \cup B \Leftrightarrow x \not\in A \land x \not\in B\)

  % Oefening 15
  \item Analoog: \(x \not\in A \setminus B \Leftrightarrow x \not\in A \lor x \in B\)

  % Oefening 16
  \needspace{5\baselineskip}
  \item Toon aan dat
    \begin{itemize}
      \item \(A \subset B \Leftrightarrow A \cup B = B \iff A \cap B = A\)
            \begin{itemize}
              \item \(A \subset B \iff A \cup B = B\)
                    \begin{itemize}
                      \item \((\Rightarrow)\) Stel \(A \subset B\). Te bewijzen: \(A \cup B = B\), ofwel \(x \in A \cup B \iff x \in B\).
                            \begin{itemize}
                              \item \((\Leftarrow)\) \(x \in B \implies x \in A \cup B\). (definitie van unie)
                              \item \((\Rightarrow)\) Stel \(x \in A \cup B\). Te bewijzen: \(x \in B\).
                                    \begin{align*}
                                      x \in A \cup B &\iff x \in A \lor x \in B \\
                                                    &\iff x \in B \lor x \in B \quad (\text{want } A \subset B \Rightarrow x \in A \implies x \in B) \\
                                                    &\iff x \in B
                                    \end{align*}
                            \end{itemize}
                      \item \((\Leftarrow)\) Stel \(A \cup B = B\). Te bewijzen: \(A \subset B\), ofwel \(x \in A \implies x \in B\).
                            \begin{align*}
                              x \in A &\implies x \in A \cup B \\
                                      &\implies x \in B
                            \end{align*}
                    \end{itemize}
              \item \(A \cup B = B \iff A \cap B = A\)
                    \begin{itemize}
                      \item \((\Rightarrow)\) Stel \(A \cup B = B\). Te bewijzen: \(A \cap B = A\), ofwel \(x \in A \cap B \iff x \in A\).
                            \begin{align*}
                              x \in A \cap B &\implies x \in A \land x \in B \\
                                             &\implies x \in A \land (x \in A \cup B) \\
                                             &\implies x \in A \land (x \in A \lor x \in B) \\
                                             &\implies x \in A
                            \end{align*}
                      \item \((\Leftarrow)\) Stel \(A \cap B = A\). Te bewijzen: \(A \cup B = B\), ofwel \(x \in A \cup B \iff x \in B\).
                            \begin{align*}
                              x \in A \cup B &\implies x \in A \lor x \in B \\
                                             &\implies (x \in A \cap B) \lor x \in B \\
                                             &\implies (x \in A \land x \in B) \lor x \in B \\
                                             &\implies x \in B \\
                            \end{align*}
                    \end{itemize}
            \end{itemize}

      \item \(A \cup (A \cap B) = A\)
            \begin{itemize}
              \item Te bewijzen: \(x \in A \cup (A \cap B) \iff x \in A\)
                    \begin{itemize}
                      \item \((\Leftarrow)\) \(x \in A \implies x \in A \cup (A \cap B)\) (definitie van unie)
                      \item \((\Rightarrow)\) \(x \in A \cup (A \cap B) \implies x \in A\)
                            \begin{align*}
                              x \in A \cup (A \cap B) &\iff x \in A \lor x \in (A \cap B) \\
                                                      &\iff x \in A \lor (x \in A \land x \in B) \\
                                                      &\iff x \in A
                            \end{align*}
                    \end{itemize}
            \end{itemize}

      \item \(A \cap (A \cup B) = A\)
            \begin{itemize}
              \item Te bewijzen: \(x \in A \cap (A \cup B) \iff x \in A\)
                    \begin{itemize}
                      \item \((\Rightarrow)\) \(x \in A \cap (A \cup B) \implies x \in A\).
                            \begin{align*}
                              x \in A \cap (A \cup B) &\implies x \in A \land x \in (A \cup B) \\
                                                     &\implies x \in A \land (x \in A \lor x \in B)) \\
                                                     &\implies x \in A
                            \end{align*}
                      \item \((\Leftarrow)\) \(x \in A \implies x \in A \cap (A \cup B)\)
                            \begin{align*}
                              x \in A &\implies x \in A \land x \in A \\
                                      &\implies x \in A \land (x \in A \lor x \in B) \\
                                      &\implies x \in A \cap (A \cup B) \\
                            \end{align*}
                    \end{itemize}
            \end{itemize}
    \end{itemize}
    
  % Oefening 17
  \item Wanneer is \(x \not\in \bigcup_{i \in I} A_i\)? \\
        Wanneer \(x \not\in A\) of de index van \(x\) niet in \(I\) zit.

  % Oefening 18
  \item Geef de betekenis in woorden van de volgende uitspraken. Zeg of ze waar of onwaar zijn. Geef de negatie in symbolen en woorden.
  \begin{itemize}
    \item \(\forall x \in \mathbb{Z}, \exists y \in \mathbb{Z} : x < y\)
          \begin{itemize}
            \item Betekenis: voor elk geheel getal \(x\) bestaat er een groter geheel getal \(y\).
            \item Waar.
            \item Negatie: \(\exists x \in \mathbb{Z}, \forall y \in \mathbb{Z} : x \geq y\)
          \end{itemize}
    \item \(\exists x \in \mathbb{Z}, \exists y \in \mathbb{N} : x > y\)
          \begin{itemize}
            \item Betekenis: er bestaat een geheel getal \(x\) dat groter is dan een natuurlijk getal \(y\).
            \item Waar.
            \item Negatie: \(\forall x \in \mathbb{Z}, \forall y \in \mathbb{N} : x \le y\)
          \end{itemize}
    \item \(\exists x \in \mathbb{Z}, \forall y \in \mathbb{Z} : x < y\)
          \begin{itemize}
            \item Betekenis: er bestaat een geheel getal \(x\) dat kleiner is dan elk geheel getal \(y\).
            \item Onwaar, dan zou dit getal ook kleiner dan zichzelf moeten zijn.
            \item Negatie: \(\forall x \in \mathbb{Z}, \exists y \in \mathbb{Z} : x \ge y\)
          \end{itemize}
    \item \(\forall \varepsilon > 0, \exists \delta > 0, \forall x \in \mathbb{R} : |x-a| < \delta \Rightarrow |f(x) - f(a)| < \varepsilon\)
          \begin{itemize}
            \item Betekenis: er bestaan een aantal getallen waarvoor geldt dat als het verschil 
                  tussen \(x\) en \(a\) kleiner is dan \(\delta\), het verschil tussen \(f(x)\) en 
                  \(f(a)\) kleiner is dan \(\varepsilon\).
            \item Waar, dit is de definitie van een limiet.
            \item Negatie: \(\exists \varepsilon > 0, \forall \delta > 0, \exists x \in \mathbb{R} : |x-a| < \delta \Rightarrow |f(x) - f(a)| \geq \varepsilon\)
          \end{itemize}
  \end{itemize}

  % Oefening 19
  \item Zij \(A = \{2,3\}\), \(B = \{4,5,6\}\) en \(C = \{a,b,c,d\}\). Geef \(A \times B\), \(B \times A\), \(A \times C\), \(C \times B\), \(A^2\), \(C \times \{a\}\).
  \begin{itemize}
    \item \(A \times B = \{(2,4), (2,5), (2,6), (3,4), (3,5), (3,6)\}\)
    \item \(B \times A = \{(4,2), (4,3), (5,2), (5,3), (6,2), (6,3)\}\)
    \item \(A \times C = \{(2,a), (2,b), (2,c), (2,d), (3,a), (3,b), (3,c), (3,d)\}\)
    \item \(C \times B = \{(a,4), (a,5), (a,6), (b,4), (b,5), (b,6), (c,4), (c,5), (c,6), (d,4), (d,5), (d,6)\}\)
    \item \(A^2 = A \times A = \{(2,2), (2,3), (3,2), (3,3)\}\)
    \item \(C \times \{a\} = \{(a,a), (b,a), (c,a), (d,a)\}\)
  \end{itemize}

  % Oefening 20
  \item Zij \(A = \{1,2,3,4\}\) en beschouw de relatie \(\mathcal{R}\): "is kleiner dan of gelijk aan" op \(A\). Geef de elementen van \(\mathcal{R}\). Geef de inverse relatie van \(\mathcal{R}\).
  \begin{itemize}
    \item \( \mathcal{R} = \{(1,1), (1,2), (1,3), (1,4), (2,2), (2,3), (2,4), (3,3), (3,4), (4,4)\} \)
    \item \(\mathcal{R}^{-1}\) bevat alle paren \((b,a)\) waarvoor \(a \leq b\), dus \((b,a)\) met \(b \geq a\): \\
    \( \mathcal{R}^{-1} = \{(1,1), (2,1), (2,2), (3,1), (3,2), (3,3), (4,1), (4,2), (4,3), (4,4)\} \)
  \end{itemize}

  % Oefening 21
  \needspace{5\baselineskip}
  \item Zij \(f : A \rightarrow B\) een functie en \(S_1, S_2 \subset A\). Bewijs dat
  \begin{enumerate}[label=\alph*)]
    \item \(f(S_1 \cup S_2) = f(S_1) \cup f(S_2)\)
    \item \(f(S_1 \cap S_2) \subset f(S_1) \cap f(S_2)\)
  \end{enumerate}
  Zoek voorbeelden die dit illustreren.
  
  
  
  \begin{tcolorbox}[title=Ter herinnering]
    Voor een functie \(f : A \rightarrow B\) en \(S \subset A \) definiëren we het beeld van \(S\)
    door \(f\) als 
    \begin{align*}
      f(S) &= \{f(s) \mid s \in S\} \\
          &= \{b \in B \mid \exists s \in S : f(s) = b\}
    \end{align*}
  \end{tcolorbox}

  \begin{enumerate}[label=\alph*)]
    \item \(f(S_1 \cup S_2) = f(S_1) \cup f(S_2)\), ofwel 
          \(b \in f(S_1 \cup S_2) \iff b \in f(S_1) \cup f(S_2)\)
          \begin{itemize}
            \item \((\Rightarrow)\) Stel \(b \in f(S_1 \cup S_2)\). Te bewijzen: \(b \in f(S_1) \cup f(S_2)\).
                  \begin{align*}
                    b \in f(S_1 \cup S_2) &\iff \exists s \in S_1 \cup S_2 : f(s) = b \\
                                          &\iff (\exists s \in S_1 : f(s) = b) \lor (\exists s \in S_2 : f(s) = b) \\
                                          &\iff b \in f(S_1) \lor b \in f(S_2) \\
                                          &\iff b \in f(S_1) \cup f(S_2)
                  \end{align*}
            \item \((\Leftarrow)\) Aangezien hierboven \(\iff\) gebruikt werd, is dit ook bewezen.
          \end{itemize}

    \item \(f(S_1 \cap S_2) \subset f(S_1) \cap f(S_2)\), ofwel
          \(b \in f(S_1 \cap S_2) \implies b \in f(S_1) \cap f(S_2)\)
          \begin{align*}
            b \in f(S_1 \cap S_2) &\iff \exists s \in S_1 \cap S_2 : f(s) = b \\
                                  &\iff (\exists s \in S_1 : f(s) = b) \land (\exists s \in S_2 : f(s) = b) \\
                                  &\iff b \in f(S_1) \land b \in f(S_2) \\
                                  &\iff b \in f(S_1) \cap f(S_2)
          \end{align*}
          
  \end{enumerate}

  % Oefening 22
  \item Zij \(f : A \rightarrow B\) een functie die niet noodzakelijk inverteerbaar is, en \(S \subset A\) en \(T, T_1, T_2 \subset B\). Bewijs dat
  \begin{enumerate}[label=\alph*)]
    \item \(f^{-1}(T_1 \cup T_2) = f^{-1}(T_1) \cup f^{-1}(T_2)\)
    \item \(f^{-1}(T_1 \cap T_2) = f^{-1}(T_1) \cap f^{-1}(T_2)\)
    \item \(f(f^{-1}(T)) \subset T\)
    \item \(f^{-1}(f(S)) \supset S\)
  \end{enumerate}
  Zoek voorbeelden die dit illustreren.

  \begin{tcolorbox}[title=Ter herinnering]
    Voor een functie \(f : A \rightarrow B\) en \(T \subset B \) definiëren we het inverse beeld
    van \(T\) onder \(f\) als
    \[ f^{-1}(T) = \{a \in A \mid f(a) \in T\} \]
  \end{tcolorbox}

  \begin{enumerate}[label=\alph*)]
    \item \(f^{-1}(T_1 \cup T_2) = f^{-1}(T_1) \cup f^{-1}(T_2)\), ofwel
          \(a \in f^{-1}(T_1 \cup T_2) \iff a \in f^{-1}(T_1) \cup f^{-1}(T_2)\)
          \begin{itemize}
            \item \((\Rightarrow)\) Stel \(a \in f^{-1}(T_1 \cup T_2)\). Te bewijzen: \(a \in f^{-1}(T_1) \cup f^{-1}(T_2)\).
                  \begin{align*}
                    a \in f^{-1}(T_1 \cup T_2) &\iff f(a) \in T_1 \cup T_2 \\
                                               &\iff f(a) \in T_1 \lor f(a) \in T_2 \\
                                               &\iff a \in f^{-1}(T_1) \lor a \in f^{-1}(T_2) \\
                                               &\iff a \in f^{-1}(T_1) \cup f^{-1}(T_2)
                  \end{align*}
                  Aangezien er \(\iff\) is gebruikt, is ook de andere richting bewezen.
          \end{itemize}

    \item \(f^{-1}(T_1 \cap T_2) = f^{-1}(T_1) \cap f^{-1}(T_2)\), ofwel
          \(a \in f^{-1}(T_1 \cap T_2) \iff a \in f^{-1}(T_1) \cap f^{-1}(T_2)\)
          \begin{itemize}
            \item \((\Rightarrow)\) Stel \(a \in f^{-1}(T_1 \cap T_2)\). Te bewijzen: \(a \in f^{-1}(T_1) \cap f^{-1}(T_2)\).
                  \begin{align*}
                    a \in f^{-1}(T_1 \cap T_2) &\iff f(a) \in T_1 \cap T_2 \\
                                               &\iff f(a) \in T_1 \land f(a) \in T_2 \\
                                               &\iff a \in f^{-1}(T_1) \land a \in f^{-1}(T_2) \\
                                               &\iff a \in f^{-1}(T_1) \cap f^{-1}(T_2)
                  \end{align*}
                  Aangezien er \(\iff\) is gebruikt, is ook de andere richting bewezen.
          \end{itemize}
          
    \item \(f(f^{-1}(T)) \subset T\), ofwel 
          \(b \in f(f^{-1}(T)) \implies b \in T\)
          \begin{align*}
            b \in f(f^{-1}(T)) &\iff \exists a \in f^{-1}(T) : f(a) = b \\
                               &\iff \exists a \in A : f(a) = b \land f(a) \in T \\
                               &\implies b \in T
          \end{align*}
    \item \(f^{-1}(f(S)) \supset S\), ofwel 
          \(s \in S \implies s \in f^{-1}(f(S))\)
          \begin{align*}
            s \in S &\implies 
          \end{align*}
  \end{enumerate}

  % Oefening 23
  \item Toon aan: \(f : A \rightarrow B\) is injectief \(\Leftrightarrow \forall b \in B : f^{-1}(b)\) bevat hoogstens één element.

  % Oefening 24
  \item
  \begin{enumerate}[label=\alph*)]
    \item Zij \(f : A \rightarrow B\). Toon aan dat \(f \circ 1_A = f = 1_B \circ f\).
    \item Toon aan dat de samenstelling van 2 injecties opnieuw een injectie is.
    \item Toon aan dat de samenstelling van 2 surjecties opnieuw een surjectie is.
  \end{enumerate}

    % Oefening 25
    \item Zij \(f(x) = \sqrt{x}\), \(g(x) = x/4\) en \(h(x) = 4x - 8\). Zoek het functievoorschrift voor:
    \begin{enumerate}[label=\alph*)]
      \item \(h \circ g \circ f = 4 \left(\frac{\sqrt{x}}{4}\right) - 8 = \sqrt{x} - 8\)
      \item \(h \circ f \circ g = 4 \left(\sqrt{\frac{x}{4}}\right) - 8 = 2 \sqrt{x} - 8\)
      \item \(g \circ h \circ f = \frac{4 \sqrt{x} - 8}{4} = \sqrt{x} - 2\)
      \item \(g \circ f \circ h = \frac{\sqrt{4x-8}}{4} = \frac{\sqrt{4(x-2)}}{4} = \frac{\sqrt{x-2}}{2}\)
      \item \(f \circ g \circ h = \sqrt{\frac{4x-8}{4}} = \sqrt{x-2}\)
      \item \(f \circ h \circ g = \sqrt{4\frac{x}{4}-8} = \sqrt{x-8}\)
    \end{enumerate}

    % Oefening 26
    \item Zij \(f(x) = x-3\), \(g(x) = \sqrt{x}\), \(h(x) = x^3\) en \(j(x) = 2x\). Schrijf de volgende functies als een samenstelling van de bovenstaande:
    \begin{enumerate}[label=\alph*)]
      \item \(\sqrt{x-3} = g \circ f\)
      \item \(2\sqrt{x} = j \circ g\)
      \item \(x^{1/4} = g \circ g\)
      \item \(4x = j \circ j\)
      \item \(\sqrt{(x-3)^3} = g \circ h \circ j\)
      \item \((2x-6)^3 = h \circ j \circ f\)
      \item \(2x-3 = f \circ j\)
      \item \(x^{3/2} = g \circ h = h \circ g\)
      \item \(x^9 = h \circ h\)
      \item \(x-6 = f \circ f\)
      \item \(2\sqrt{x-3} = j \circ g \circ f\)
      \item \(\sqrt{x^3-3} = g \circ f \circ h\)
    \end{enumerate}

    % Oefening 27
    \item Toon aan voor inverteerbare functies \(f\) en \(g\):
    \begin{enumerate}[label=\alph*)]
      \item \((f^{-1})^{-1} = f\)
      \item \((g \circ f)^{-1} = f^{-1} \circ g^{-1}\)
    \end{enumerate}

    % Oefening 28
    \item Onderzoek of volgende functies inverteerbaar zijn. Zo ja, bepaal de inverse functies. Zo nee, definieer een bijectie \(\hat{f}\) met hetzelfde voorschrift als \(f\) en bepaal \((\hat{f})^{-1}\).
    \begin{enumerate}[label=\alph*)]
      \item \(f : \mathbb{R} \to \mathbb{R} : x \mapsto |x|\): \\
            \(\hat{f} : \mathbb{R}^+ \to \mathbb{R}^+ : x \mapsto |x|\) \\
            \(\hat{f}^{-1} : \mathbb{R}^+ \to \mathbb{R}^+ : x \mapsto x\)
      \item \(f : \mathbb{R} \to \mathbb{R} : x \mapsto x+1\): \\
            \(f^{-1} : \mathbb{R} \to \mathbb{R} : x \mapsto x - 1\)
      \item \(f : \mathbb{R} \to \mathbb{R} : x \mapsto 2x+3\): \\
            \(f^{-1} : \mathbb{R} \to \mathbb{R} : x \mapsto \frac{x-3}{2}\)
      \item \(f : \mathbb{R}^+ \to \mathbb{R}^+ : x \mapsto \sqrt{x}\): \\
            \(f^{-1} : \mathbb{R}^+ \to \mathbb{R}^+ : x \mapsto x^2\)
      \item \(f : \mathbb{R} \to \mathbb{R} : x \mapsto \sqrt[3]{2x}+2\): \\
            \(f^{-1} : \mathbb{R}^+ \to \mathbb{R}^+ : x \mapsto \frac{(x-2)^3}{2}\)
      \item \(f : \mathbb{R} \to \mathbb{R} : x \mapsto 1\): \\
            \(\hat{f} : {1} \to \mathbb{R} : x \mapsto 1\) \\
            \(\hat{f}^{-1} : {1} \to {1} : x \mapsto x\)
      \item \(f : \mathbb{R}_0 \to \mathbb{R} : x \mapsto \frac{2x-3}{x}\): \\
            \(f^{-1} : \mathbb{R}^+ \to \mathbb{R}^+ : x \mapsto - \frac{3}{y - 2} = \frac{3}{2-y}\)
      \item \(f : \mathbb{R} \to \mathbb{R} : x \mapsto \sin x\): \\
            \(\hat{f} : [-\frac{\pi}{2},\frac{\pi}{2}] \to [-1,1] : x \mapsto \sin x\) \\
            \(\hat{f}^{-1} : [-1,1] \to [-\frac{\pi}{2},\frac{\pi}{2}] : x \mapsto \arcsin x\)
    \end{enumerate}

    % Oefening 29
    \item Zij \(h : \mathbb{Z} \times \mathbb{Z} \to \mathbb{Z} : h(x,y) = 2x+3y\). Bepaal het beeld van \(h\). Is \(h\) injectief? Surjectief?
          \begin{itemize}
            \item Beeld: \(\mathbb{Z}\), voor elke \(x, y \in \mathbb{Z}\) bestaat er een \(z \in \mathbb{Z}\)
            \item Niet injectief, want meerdere combinaties van \(x\) en \(y\) kunnen dezelfde \(z\) opleveren.
            \item Wel surjectief, want voor elke \(z \in \mathbb{Z}\) bestaat er een \(x, y \in \mathbb{Z}\) zodat \(z = 2x+3y\).
          \end{itemize}

    % Oefening 30
    \item Bewijs: \(f(S_1 \cap S_2) = f(S_1) \cap f(S_2)\) als \(f\) injectief is (zie oef. 21). Geef een voorbeeld van een functie waarbij \(f(S_1 \cap S_2) \neq f(S_1) \cap f(S_2)\).

    % Oefening 31
    \item Bepaal of volgende functies injectief zijn. Geef hun beeld.
    \begin{enumerate}[label=\alph*)]
      \item \(f : \mathbb{Z} \to \mathbb{Z} : x \mapsto 2x+1\)
            \begin{itemize}
              \item Injectief: er is hoogstens 1 getal \(x\) voor elk \(z \in \mathbb{Z}\) zodat \(z = 2x+1\)
              \item Beeld: \{z | z is een oneven geheel getal\}
            \end{itemize}
      \item \(f : \mathbb{Q} \to \mathbb{Q} : x \mapsto 2x+1\)
            \begin{itemize}
              \item Injectief: er is hoogstens 1 getal \(x\) voor elk \(z \in \mathbb{Q}\) zodat \(z = 2x+1\)
              \item Beeld: \{z | z is een oneven rationaal getal\}
            \end{itemize}
      \item \(f : \mathbb{Z} \to \mathbb{Z} : x \mapsto x^3-x\)
            \begin{itemize}
              \item Injectief: voor elke \(z \in \mathbb{Z}\) is er precies één \(x \in \mathbb{Z}\) zodat \(z = x^3 - x\)
              \item Beeld: \{z | z is een geheel getal met \(z = x^3 - x\)\}
            \end{itemize}
      \item \(f : \mathbb{R} \to \mathbb{R} : x \mapsto e^x\)
            \begin{itemize}
              \item Injectief: voor elke \(z \in \mathbb{R}\) is er precies één \(x \in \mathbb{R}\) zodat \(z = e^x\)
              \item Beeld: \{z | z is een macht van \(e\)\}
            \end{itemize}
      \item \(f : [-\frac{\pi}{2},\frac{\pi}{2}] \to \mathbb{R} : x \mapsto \sin x\)
            \begin{itemize}
              \item Injectief: voor elke \(z \in [-1,1]\) is er precies één \(x \in [-\frac{\pi}{2},\frac{\pi}{2}]\) zodat \(z = \sin x\)
              \item Beeld: [-1,1]
            \end{itemize}
    \end{enumerate}
\end{enumerate}
\end{document}
